\documentclass[10pt,a4paper]{article}
\usepackage[utf8]{inputenc}
\usepackage{amsmath}
\usepackage{amsfonts}
\usepackage{amssymb}
\usepackage{graphicx}
\graphicspath{ {../img/} }
\author{Gradey Cullins}
\title{Assignment 4}
\begin{document}
\maketitle

\section*{Explanation of code}
The entry point of my Matlab code is \textbf{main.m}. \textbf{main.m} is structured such that problems 1-3 are solved in successive order. The Newton Method implementation of problem 1 is contained in the function \textbf{newt\_code\_1.m}. The Newton Method implementation of problem 2 is contained in the function \textbf{newt\_hard.m}. The Newton Method implementation of problem 3 is contained in the function \textbf{newt\_loran.m}. The other functions \textbf{my\_func.m}, \textbf{hard\_funcs.m}, and \textbf{loran\_funcs.m} are used to verify my solutions by using Matlab's builtin \textit{fsolve} function.

\section*{1}

My experiment involved running the Newton Method five different times with different starting points. Beginning with [1, 0.1], the starting point for each iteration was an order of magnitude larger in the case of $x_1$ and smaller in the case of $x_2$, than the previous iteration's starting point. \\

\noindent
$[1, 0.1]$ converged after 6 iterations \\
$[10, 0.01]$ converged after 13 iterations \\
$[100, 0.001]$ converged after 19 iterations \\
$[1000, 0.0001]$ converged after 25 iterations \\
$[10000, 0.00001]$ converged after 30 iterations \\

\section*{2}

I first ran my Newton Method implementation on this series of questions. The method terminated after 8 iterations and with values: \\

\noindent
$ x_1 = 1.52e+00, x_2 = 1.37e-01 $ \\

\noindent
While my method supposedly \lq converges\rq, I investigated further by solving for $ x_1 $ and $ x_2 $ by hand: \\

\noindent
$ f_1(x_1,x_2) = x_1^2 + x_2^2 - 2 $ \\
$ f_2(x_1,x_2) = exp(x_1 - 1) - x_2^2 - 2 $ \\
$\rightarrow  f_1 - f_2 = x_1^2 + exp(x_1 - 1) - 4 = 0 $ \\
$\rightarrow  x_1^2 + exp(x_1 - 1) = 4 \rightarrow x_1 \approx -1.99, 1.52 $  \\
substitute $x_1$ into $f_2$  $ \rightarrow exp(-1.99 - 1) - x_2^2 - 2 = 0  $ \\
\indent $ \rightarrow x_2 \approx \pm 1.39i $ \\
$ \rightarrow exp(1.52 - 1) - x_2^2 - 2 = 0 $ \\
\indent $ \rightarrow x_2 \approx \pm 0.56i $ \\

\noindent
Because the solutions are complex, the straightforward  conclusion is that this series of equations does not have a solution. \\

\noindent
This conclusion is supported by the results of calling matlab's fsolve function at starting points: \\
$ x_1 = 1.1, x_2 = 1.1 $ \\
$ x_1 = 2, x_2 = 0.5 $ \\
$ x_1 = 3, x_2 = 5 $ \\

\noindent
All calls to fsolve reported inability to find a solution. In other words, these starting values do not appear to help in solving this difficult system of equations. 

\section*{3}

Running the solution for 200 different starting points in the range $ [400, 400] - [600, 600] $ yielded three distinct points of convergence (solutions). Those solutions are listed below: \\

\noindent
$ x_1 = 2.542211e+02, x_2 = 2.193070e+02 $ \\
$ x_1 = 7.403288e+02, x_2 = 9.068259e+02 $ \\
$ x_1 = -1.932946e+02, x_2 = 6.656490e+01 $ \\

\noindent
As shown above, there is a total of three solutions. \\

\noindent
Running the problem section of the included matlab code will display the full list of starting values and their corresponding points of convergence.


\end{document}
